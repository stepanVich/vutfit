\documentclass[11pt,titlepage,a4paper]{article}

\usepackage[utf8]{inputenc}
\usepackage[T1]{fontenc}
\usepackage[czech]{babel}
\usepackage[left=2cm,top=3cm,text={17cm,24cm}]{geometry}
\usepackage{multirow}
\bibliographystyle{czechiso}
\begin{document}



\begin{titlepage}

\begin{center}

\textsc{\Huge{Vysoké učení technické v~Brně} \\[3mm] \huge{Fakulta informačních technologií}}\\ 
\vspace{\stretch{0.382}}
\huge{Typografie a publikování - 4. projekt}\\[0mm]
\Huge{Bibliografické citace}
\vspace{\stretch{0.618}}


\Large{13.duben 2018 \hfill
Štěpán Vích}
\end{center}
\end{titlepage}


\section{Úvodní slovo}

V tomto článku bych chtěl poukázat na výhody sázecího systému \LaTeX oproti klasickým WISYWYG editorům typu MS Word, Libre Office a dalším. Budou zde citace dokumentů a knih, které sami o sobě zvyšují důvěryhodnost tohoto textu. 


\section{Proč \LaTeX}

Nedávno jsem četl články v anglickém magazínu Science. Byly to články \cite{science1} a \cite{science2}. Tyto články však nebyly vysázeny v systému MS Word, nýbrž v pokročilých sázecích systémech. Jedním z těchto systémů je i systém \LaTeX. Jeho složitost může začátečníky odradit, ale po zvládnutí základů, je psaní větších dokumentů systematyčtější a rychlejší. Další nespornou výhodou je lepší vzhled dokumentů. Ten je dosažen lepším systémem sázení znaků.

\section{Učební texty}

Pokud jste v pozici začátečníka v systému \LaTeX, máte nepřeberné množství materiálů, které vám s učením pomohou. Pro rodilé mluvčí je určena kniha \cite{rybicka}. Pro anglicky mluvící je zde kniha \cite{latex-beginners-guide}. Dále je zde mnoho online dokumentů pro ty, kteří nechtějí nebo nemají čas na tištěné knihy a to např. \cite{dokument1}, \cite{dokument2} a \cite{dokument3}.

O \LaTeX u se také pořádají konference. To, že se některé nezúčastníte nebude vadit. Z většiny konferencí se vydávají sborníky a veřejně se zpřístupní. Příkladem takového sborníku je tento \cite{sbornik}


\section{Příklady prací}

U studentů je \LaTeX oblíben natolik, že v něm i tvoří bakalářské, diplomové nebo disertační práce. Mezi takovéto práce patří například diplomová práce \cite{diplomka1} nebo \cite{diplomka2}. Doufám, že i vám se s \LaTeX em bude dařit a oblíbíte si ho jako já. S přibývajicími zkušenostmi zjistíte, že psaní dokumentů vám jde rychleji a výsledný vzhled je mnohem lepší.


\newpage
\bibliography{reference}


\end{document}

